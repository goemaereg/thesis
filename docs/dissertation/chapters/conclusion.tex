\chapter{Conclusion} \label{ch:conclusion}

In this work we proposed an enhancement to the \acrlong{wsol} task to evaluate localization of multiple object instances. We have argued that none of existing \acrshort{wsol} evaluation metrics are suited to quantitatively measure multi-instance localization performance. We have then formally defined a new evaluation methodology and metrics to measure multi-instance object localization accuracy for a CAM-based family of localization methods, requiring only standard classification models.

We validated our multi-instance localization methodology quantitatively using CAM-based methods (CAM, Grad-CAM, Grad-CAM++, MinMaxCAM, Score-CAM) for well-known CNN models (standard VGG16 and ResNet-50, and modified VGG16-GAP) and datasets (synthetic and ImageNet), measuring precision and recall for datasets with ground truth bounding box and segmentation mask annotations.

We demonstrated that multi-instance localization recall metrics are correlated with the number of object instances in images, and that this is valid for the assessed network architectures and regularization techniques. We showed that classification and localization accuracy are correlated while training a model, and established the importance of using a trained network that has converged to reliably measure localization performance.

We proposed a novel iterative localization approach using previously predicted object localization information as input to improve localization recall. We demonstrated that localization recall can be improved using the right strategy to combine iterative predictions, and that this comes at the cost of a lower precision.

With these conclusions we answered the research question, i.e., we can use CAM-based methods to evaluate and improve localization of multiple instances of the same class within images for the \acrshort{mwsol} task.