\chapter{Methodology}

\section{General localization approach}
\begin{itemize}
    \item Based on WSOL CAM methods.
    \item explain localization pipeline
    \item  \begin{itemize}
        \item classification training to learn discriminative features
        \item features scoremap computation (CAM)
        \item localization based on thresholded CAM
        \item evaluation of localization accuracy 
    \end{itemize}
\end{itemize}

\section{Setup}

\subsection{Networks}
\begin{itemize}
    \item VGG16 (used in most papers)
    \item ResNet50 (used in MinMaxCAM, more recent than VGG16)=
\end{itemize}

\subsection{Datasets}
\subsubsection{synthetic dataset}
Propose to use synthetic dataset to limit computational requirements and have control over image structure and ground truth data.

Dataset inspiration in paper: Quantifying Explainability of Saliency Methods in Deep Neural Networks with a Synthetic Dataset (Erico Tjoa, Guan Cuntai)

\begin{itemize}
    \item Easier to interprete
    \item Use segmentation masks for fine-grained localization evaluation
    \item Image size 512x512 large enough for chosen network architectures
\end{itemize}

\subsubsection{ImageNet-1k} 
\begin{itemize}
    \item Used in most CAM papers (e.g. CAM, MinMaxCAM, Grad-CAM, ScoreCAM)
    \item Datasets with single category GT and multiple instances per image available for object localization task (ILSVRC 2012)
\end{itemize}

\subsection{CAM WSOL methods}
Explain each method and the reason why they are used.
\begin{itemize}
    \item CAM: baseline method
    \item MinMaxCAM: regularization of common object regions and background
    \item GradCAM: Generalization of CAM that doesn’t require a GAP layer in the architecture
    \item ScoreCAM: Shows more focus on object instances. Also seems to do a better job in capturing the features of multiple instances in the explanation map than GradCAM.
\end{itemize}

\section{Evaluation metrics}
\begin{itemize}
    \item MaxBoxAccV3: modified version of MaxBoxAccV2 (Choe et al.,2020b) to support multiple instance WSOL
    \item PxAP (pixel average precision) when ground truth segmentation masks are available.
\end{itemize}

\section{Localization improvement: Iterative bounding box extraction}
\begin{itemize} 
    \item Feed image to model and compute feature activation maps (CAMS)
    \item Extract bounding boxes from binarized CAMS
    \item Mask bounding box areas in image (e.g. with noise, zero values) 
    \item Repeat until some stop criterion
\end{itemize}
Variant: Mask image with binarized CAMs.

Possible stop criteria: 
\begin{itemize}
    \item No new bounding boxes are found
    \item Predefined minimum CAM threshold is reached
    \item Predefined drop in classification score is reached
    \item Predefined number of iterations is reached.
\end{itemize}
