\chapter*{Samenvatting}

Weakly-supervised object localization (WSOL) heeft aan populariteit gewonnen wegens de eigenschap dat netwerkmodellen, getrained met beeldbestanden waarvoor enkel class labels op beeldniveau beschikbaar zijn, kunnen gebruikt worden voor lokalisatie van voorwerpen in beelden. De class activation mapping (CAM) familie van WSOL methodes focust op verschillende strategieën om voorwerpen te beschrijven en te lokaliseren. Deze methodes gebruiken in veel gevallen verschillende metrieken om de lokalisatie performantie te meten. Deze metrieken zijn echter niet geschikt voor het meten van de locatie van meerdere objecten in beelden.

In dit werk introduceren we een evaluatieprotocol voor multiple-instance object lokalisatie. Dit protocol breidt bestaande metrieken uit om het meten van de locatie van meerdere objecten van dezelfde class mogelijk te maken. Deze methode maakt gebruik van de standaard classificatie taak en vereist geen wijzigingen van netwerken.

Gebruik makend van het voorgestelde evaluatieprotocol voor de CAM-gebaseerde WSOL taak, stellen we vast dat de lokalisatie performantie van CAM-methodes, geëvalueerd op netwerk modellen, daalt naarmate er meer objecten aanwezig zijn in beelden. Dit toont aan dat ons evaluatieprotocol de lokalisatie nauwkeurigheid voor meerdere objecten van dezelfde class in beelden kwantitatief kan meten. De geëvalueerde CAM-methodes en netwerk modellen dienen verder als basis voor het evalueren van verbeterde lokalisatiemethodes.

Om de performantie van multiple-instance lokalisatie te verbeteren, introduceren we een iteratieve lokalisatiemethode. We gebruiken lokalisatieinformatie van objecten, voorspeld in beelden tijdens vorige iteraties, om beelden te maskeren met de locatie van de reeds gevonden objecten, en om zo het lokaliseren van andere objecten mogelijk te maken. We tonen aan dat, gebruik makende van verschillende strategiëen om object locaties gevonden tijdens verschillende iteraties samen te voegen, de lokalisatie recall verbeterd kan worden. We tonen ook aan dat de verbetering in recall in vele gevallen ten koste gaat van de precision, en dus meer en meer voorspelde locaties foutief objecten aanduiden.