\chapter*{Samenvatting}

Weakly-supervised object localization (WSOL) heeft aan populariteit gewonnen wegens de eigenschap dat netwerkmodellen, getrained met beeldbestanden waarvoor enkel class labels op beeldniveau beschikbaar zijn, kunnen gebruikt worden voor lokalisatie van voorwerpen in beelden. De class activation mapping (CAM) familie van lokalisatiemethodes, focust op verschillende strategieën om voorwerpen te beschrijven en te lokaliseren. Deze methodes gebruiken vaak verschillende metrieken om de performantie van lokalisatiemethodes te meten. Deze metrieken zijn echter geen goede maat voor het meten van de locatie van meerdere object instanties in beelden.

In dit werk stellen we een multiple-instance object lokalisatie evaluatieprotocol voor. Dit protocol breidt bestaande metrieken uit om het meten van de locatie van meerdere object instanties mogelijk te maken. Deze methode maakt gebruik van de standaard classificatietaak en vereist geen wijzigingen van netwerken.

Gebruik makend van ons evaluatieprotocol voor CAM-gebaseerde lokalisatie, stellen we vast dat de performantie van de lokalisatiemethodes voor de geëvalueerde modellen daalt naarmate er meer object instanties aanwezig zijn in beelden. Dit toont aan dat onze metrieken de nauwkeurigheid van object lokalisatie kwantitatief kan meten voor beelden met meerdere object instanties. We gebruiken de geëvalueerde modellen als basis voor het evalueren van verbeterde lokalisatiemethodes.

Om de performantie van multiple-instance lokalisatie te verbeteren, introduceren we een iteratieve lokalisatie methode. Hiervoor gebruiken we lokalisatie informatie van objecten  voorspeld in beelden tijdens vorige iteraties, om beelden te maskeren en voor het lokaliseren van nieuwe objecten. We tonen aan dat met de manier waarop we lokalisatie informatie uit verschillende iteraties samenvoegen, de recall van lokalisatie behoorlijk kan verbeterd worden. Dit gaat in vele gevallen ten koste van de precision, waarbij meer en meer voorspelde locaties foutief objecten aanduiden.