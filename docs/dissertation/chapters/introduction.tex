\chapter{Introduction}

\acrfull{wsol} is an object recognition task that is best explained by positioning it in the landscape of object recognition tasks. Object recognition is a generic term covering a set of computer vision tasks for identifying objects in digital images: Image classification, object localization, object detection and semantic segmentation. 
\\\\
Image classification is a task that assigns a single label to an image. So, an image classification algorithm takes as input an image with one or more objects and returns a class label. For example, given an image depicting an animal, an image classification model would return one of the labels "cat", "dog", "horse", etc. 
\\\\
An object localization algorithm deals with image classification and localizing objects by generating bounding boxes that surround the objects. Each bounding box represents the location of a detected object. A bounding box is typically represented by two points: the top-left and bottom-right coordinates of the box. An object localization algorithm doesn't return a label for a detected object. There is only one label per image as the result of classification. As a consequence, object localization only recognizes objects of the same class.
\\\\
Object detection is the task of localizing objects in an image with bounding boxes and assigning a label for each detected object. Hence, an object detection model takes as input an image with one or more objects and returns a set of bounding boxes and corresponding class labels.
\\\\
Semantic segmentation assigns a class label to each pixel in an image. This task can't localize different instances of the same class in an image, but returns a single mask including all instances of the same class.
\\\\
State-of-the-art object detection models are implemented using deep neural networks that are trained using supervised machine learning. Supervised learning for object detection requires that each object in an image must be assigned a class label and a bounding box that localizes the object in the image. This human labeling is a costly and error-prone task. Object localization only deals with recognition of a single object class in an image and can be seen as a simplified object detection task. It requires only a single image-level class label and bounding boxes for each object of that class in the image. The goal of the object localization task is to cover the full extent of an object in a digital image.
\\\\
\acrshort{wsol} is an object localization task that trains an object localization model by only using image-level labels. Hence the term weakly: Training requires image class labels but no localization bounding boxes. Because costly human labeling of object locations is not required, \acrshort{wsol} research have gained significant momentum \cite{zhou2016cvpr, selvaraju2017grad, chattopadhay2018grad, wang2021minmaxcam, wang2020score, choe2020evaluating}.
\\\\
The baseline \acrshort{wsol} method is \acrfull{cam} \cite{zhou2016cvpr}. This method uses feature that are activated on the most discriminative parts of an object to localize an object of a specific class. This focus on the mosts discriminative parts is a limitation of the \acrshort{cam} method, as it doesn't cover the complete object. New \acrshort{wsol} research \cite{selvaraju2017grad, chattopadhay2018grad, wang2021minmaxcam, wang2020score, choe2020evaluating} focuses on overcoming the limitations of the baseline \acrshort{cam} method. As the CAM-family of methods represents a main body of research for object localization, we will use a specific list of CAM methods in this project. The relevant methods will be discussed in chapter \ref{ch:related_work}. 
\\\\
Certain CAM-related techniques are mainly used for explainability, i.e. they are used to visually explain why a model predicts a certain class label for an image. Some of these methods indicate the ability to detect multiple object of the same class in an image and show promising results in qualitative experiment results \cite{wang2020score}.
\\\\
Many \acrshort{cam} papers report performance improvements over the baseline \acrshort{cam} method. \cite{choe2020evaluating} criticizes that \acrshort{cam} methods lack a unified definition of the \acrshort{wsol} task and proposes a new evaluation protocol localization. A problem is that \acrshort{wsol} has not been tested for localization of multiple object instances of the same class and current \acrshort{wsol} evaluation metrics are not sufficient to measure multiple instance localization. Given that multiple-instance localization hasn't been tested, it is interesting to evaluate it. 
\\\\
The research question for us is whether we can use CAM-methods to evaluate and improve localization of multiple instances of the same class within images for the \acrshort{wsol} task.
\\
Therefor, we propose an evaluation protocol for localization of multiple-instances of the same class by enhancing existing evaluation metrics \cite{choe2020evaluating} to measure multiple-instance localization performance. We will call this method \acrfull{mwsol}. We then benchmark existing CAM methods for multiple instance localization. Finally, we investigate improvements for the \acrshort{mwsol} task using iterative bounding box extraction method. Bounding boxes of objects localized in previous iterations, are used to mask their location in images which are then used to find the location of objects missed during previous iterations.\\
